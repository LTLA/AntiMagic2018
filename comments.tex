\documentclass[10pt,letterpaper]{article}
\usepackage[top=0.85in,left=2.75in,footskip=0.75in,marginparwidth=2in]{geometry}

% use Unicode characters - try changing the option if you run into troubles with special characters (e.g. umlauts)
\usepackage[utf8]{inputenc}

% clean citations
\usepackage{cite}

% hyperref makes references clicky. use \url{www.example.com} or \href{www.example.com}{description} to add a clicky url
\usepackage{nameref,hyperref}

% line numbers
\usepackage[right]{lineno}

% improves typesetting in LaTeX
\usepackage{microtype}
\DisableLigatures[f]{encoding = *, family = * }

% text layout - change as needed
\raggedright
\setlength{\parindent}{0.5cm}
\textwidth 5.25in 
\textheight 8.75in

% Remove % for double line spacing
%\usepackage{setspace} 
%\doublespacing

% use adjustwidth environment to exceed text width (see examples in text)
\usepackage{changepage}

% adjust caption style
\usepackage[aboveskip=1pt,labelfont=bf,labelsep=period,singlelinecheck=off]{caption}

% remove brackets from references
\makeatletter
\renewcommand{\@biblabel}[1]{\quad#1.}
\makeatother

% headrule, footrule and page numbers
\usepackage{lastpage,fancyhdr,graphicx}
\usepackage{epstopdf}
\pagestyle{myheadings}
\pagestyle{fancy}
\fancyhf{}
\rfoot{\thepage/\pageref{LastPage}}
\renewcommand{\footrule}{\hrule height 2pt \vspace{2mm}}
\fancyheadoffset[L]{2.25in}
\fancyfootoffset[L]{2.25in}

% use \textcolor{color}{text} for colored text (e.g. highlight to-do areas)
\usepackage{color}

% define custom colors (this one is for figure captions)
\definecolor{Gray}{gray}{.25}

% this is required to include graphics
\usepackage{graphicx}

% use if you want to put caption to the side of the figure - see example in text
\usepackage{sidecap}

% use for have text wrap around figures
\usepackage{wrapfig}
\usepackage[pscoord]{eso-pic}
\usepackage[fulladjust]{marginnote}
\reversemarginpar

% Adding multirow.
\usepackage{multirow}

% Other required things:
\usepackage{color}
\usepackage{subcaption}
\captionsetup[subfigure]{justification=centering}

% document begins here
\begin{document}
\vspace*{0.35in}

% title goes here:
\begin{flushleft}
{\Large
    \textbf\newline{Too much magic: avoiding illusions in single cell RNA sequencing data analysis}
}
\newline

% authors go here:
%\\
Timothy Curtis\textsuperscript{1,*}
\\
\bigskip
\bf{1} Diagon Institute of Computational Biology, London, United Kingdom \\
\bigskip

\end{flushleft}

A new imputation method ``MAGIC" \cite{vandijk2018recovering} has recently been proposed for analyzing single-cell RNA sequencing (scRNA-seq) data.
This method involves constructing an affinity matrix based on the distances between cells in high-dimensional space;
applying a diffusion process to strengthen similarities between cells due to existing trends in the data;
and multiplying the exponentiated affinity matrix with the original count matrix to obtain denoised expression values for downstream analysis.
The authors claim that the use of MAGIC removes technical noise and improves the biological signal in scRNA-seq data.
Indeed, the noisy nature of scRNA-seq data has been documented \cite{grun2015design} and many computational methods have been developed to address this problem \cite{bacher2016design}.
Unfortunately - as we will argue below - using MAGIC is not the solution.

The raison d'\^etre of the MAGIC procedure is the use of the diffusion process to reinforce existing trends.
The affinity matrix is exponentiated to represent the probability of one cell reaching another via a number of intermediate cells.
This procedure systematically increases the affinities between pairs of cells that are connected by many close neighbours.
Conversely, affinities are downweighted between pairs of cells in sparse regions of the high dimensional space, where similarities are more likely to be caused by noise or outliers.
By performing diffusion, MAGIC can achieve aggressive denoising compared to other methods like $k$-NN imputation.
However, this also runs the risk of introducing spurious structure in the denoised expression matrix.

This effect is best demonstrated with a simple simulation.
We applied MAGIC to a simulated matrix of Poisson-distributed counts that contained no structure.
Upon performing a principal components analysis (PCA) on the denoised expression matrix, we observed that the first principal component (PC) explained much more variance (Figure~\ref{fig:nostructure}a) than expected under randomness (Figure~\ref{fig:nostructure}b).
This is a manifestation of spurious structure whereby an artificial ``trajectory'' along the first PC is manufactured from nothing.
Increasing the diffusion time exacerbated the consolidation of variance into the first PC (Figure~\ref{fig:nostructure}c), consistent with self-reinforcement of the diffusion process.
While the creation of artificial structure is obviously undesirable, the interpretation of denoised expression values at the gene level is even more compromised.
After running MAGIC, genes that were truly independent now appear to be strongly co-regulated (Figure~\ref{fig:nostructure}d), whereas no such effect is observed on the original data (Figure~\ref{fig:nostructure}e).

\begin{figure}[btp]
\centering
\begin{subfigure}[b]{0.49\textwidth}
    \includegraphics[width=\textwidth,trim=0mm 5mm 0mm 15mm,clip,page=1]{pics/nostructure_with_magic.pdf}
    \caption{}
\end{subfigure}
\begin{subfigure}[b]{0.49\textwidth}
    \includegraphics[width=\textwidth,trim=0mm 5mm 0mm 15mm,clip,page=1]{pics/nostructure_without_magic.pdf}
    \caption{}
\end{subfigure}
\begin{subfigure}[b]{\textwidth}
    \includegraphics[width=\textwidth,trim=0mm 5mm 0mm 15mm,clip]{pics/nostructure_time.pdf}
    \caption{}
\end{subfigure}
\begin{subfigure}[b]{0.49\textwidth}
    \includegraphics[width=\textwidth,trim=0mm 5mm 0mm 15mm,clip,page=2]{pics/nostructure_with_magic.pdf}
    \caption{}
\end{subfigure}
\begin{subfigure}[b]{0.49\textwidth}
    \includegraphics[width=\textwidth,trim=0mm 5mm 0mm 15mm,clip,page=2]{pics/nostructure_without_magic.pdf}
    \caption{}
\end{subfigure}
\caption{Effects of applying MAGIC to a simulated dataset containing independently sampled Poisson counts with no structure.
(a) Plot of the first two PCs after performing PCA on the log-transformed denoised expression matrix.
Each point represents a cell and the percentage of variance explained by each PC is shown in parentheses.
(b) Same as (a) but on the log-transformed normalized counts without running MAGIC.
(c) Effect of diffusion time on the percentage of variance explained by the first PC.
(d) Log-denoised expression values of the two genes with the highest pairwise correlation, after running MAGIC.
Each point represents a cell.
(e) Same as (d) but without running MAGIC.}
\label{fig:nostructure}
\end{figure}

One might dismiss the previous simulation as being somewhat pathological.
After all, there are no biological systems that exhibit no structure whatsoever.
Perhaps a more pertinent question would be: how does MAGIC perform in the presence of genuine structure? 
To explore this, we performed another simulation involving four equally sized subpopulations of cells, denoted here as $A$, $B$, $C$ and $D$.
The primary separation occurs between $\{A, B\}$ and $\{C, D\}$, with a secondary separation between $A/B$ and $C/D$.
Application of MAGIC eliminated the secondary separation (Figure~\ref{fig:fourclusters}a), merging $A$ with $B$ and $C$ with $D$.
This is again a consequence of the diffusion process, where weak similarities are increasingly strengthened over diffusion time.
Specifically, the affinities between cells in $A$ and $B$ (or $C$ and $D$) increase upon exponentiation,
resulting in greater sharing of information between those pairs of subpopulations during the denoising step.
This squeezes the populations together and misrepresents the data by eliminating structure that is otherwise clearly visible without MAGIC (Figure~\ref{fig:fourclusters}b).

\begin{figure}[btp]
\centering
\begin{subfigure}[b]{0.49\textwidth}
    \includegraphics[width=\textwidth,page=1]{pics/minor_with_magic.pdf}
    \caption{}
\end{subfigure}
\begin{subfigure}[b]{0.49\textwidth}
    \includegraphics[width=\textwidth,page=1]{pics/minor_without_magic.pdf}
    \caption{}
\end{subfigure}
\caption{Plot of the first two PCs (a) after running MAGIC and (b) without running MAGIC, in a simulation containing four well-separated subpopulations of cells.
Each point represents a cell and is coloured green (subpopulations $A$ and $C$) or orange ($B$ or $D$).
$A$ and $B$ are represented by open circles while $C$ and $D$ are represented by crosses.
The percentage of variance explained by each PC is shown in parentheses.}
\label{fig:fourclusters}
\end{figure}

We also inspected the gene-wise denoised expression profiles in another simulation involving two well-separated subpopulations.
As expected, we observed strong separation between the two subpopulations regardless of whether MAGIC was used (Figure~\ref{fig:twoclusters}).
However, MAGIC also introduced spurious differences in expression between the two subpopulations among the non-DE genes (Figure~\ref{fig:twoclusters}a).
This is a consequence of sharing information between neighbouring cells, which squeezes the expression values for all cells towards their subpopulation means.
Estimation uncertainty will yield a different sample mean for each subpopulation, even for non-DE genes.
These small differences are amplified by aggressive denoising, manifesting as in strong differences in the distribution of expression values betweeen subpopulations.
One could argue that this spurious DE pattern is not of concern as the log-fold changes for the genes involved are relatively small (less than 0.2 in most cases).
However, this attitude is not consistent with the intention behind the use of denoising in the first place, 
i.e., to recover subtle biological changes that would otherwise be masked by technical noise.

\begin{figure}[btp]
\centering
\begin{subfigure}[b]{0.49\textwidth}
    \includegraphics[width=\textwidth,page=1]{pics/clusters_with_magic.png}
    \caption{}
\end{subfigure}
\begin{subfigure}[b]{0.49\textwidth}
    \includegraphics[width=\textwidth,page=1]{pics/clusters_without_magic.png}
    \caption{}
\end{subfigure}
\caption{Heatmap of log$_2$-expression values (a) after running MAGIC and (b) without running MAGIC, in a simulation containing two well-separated subpopulations of cells.
Each column represents a cell and is labelled according to the subpopulation (i.e., cluster) to which it belongs.
Each row represents a gene that is sorted by log-fold change betwee subpopulations and labelled according to whether it is truly DE.
All log-expression values are mean-centered along each row and capped at 0.2 for visibility.}
\label{fig:twoclusters}
\end{figure}

Here, we have provided three simple simulation scenarios in which the application of MAGIC yields a distorted representation of the data.
This is primarily driven by the use of data diffusion, though some of these issues may generalize to other denoising or imputation approaches.
We recommend treating results generated by MAGIC with some caution, especially if they are not reproducible with conventional analysis strategies.

\bibliography{ref}
\bibliographystyle{unsrt}


\end{document}
